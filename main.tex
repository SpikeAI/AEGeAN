% Credits : COLING 2018 style sheets
% Contact: zhu2048@gmail.com & liuzy@tsinghua.edu.cn
%% Based on the style files for COLING-2016, which were, in turn,
%% Based on the style files for COLING-2014, which were, in turn,
%% Based on the style files for ACL-2014, which were, in turn,
%% Based on the style files for ACL-2013, which were, in turn,
%% Based on the style files for ACL-2012, which were, in turn,
%% based on the style files for ACL-2011, which were, in turn, 
%% based on the style files for ACL-2010, which were, in turn, 
%% based on the style files for ACL-IJCNLP-2009, which were, in turn,
%% based on the style files for EACL-2009 and IJCNLP-2008...

%% Based on the style files for EACL 2006 by 
%%e.agirre@ehu.es or Sergi.Balari@uab.es
%% and that of ACL 08 by Joakim Nivre and Noah Smith

\documentclass[11pt]{article}
\usepackage{coling2018}
\usepackage{times}
\usepackage{url}
\usepackage{latexsym}
\usepackage[utf8]{inputenc} 
\usepackage[T1]{fontenc}
\usepackage[french]{babel} 
\usepackage{xcolor}
\usepackage{graphicx}
\usepackage{todonotes}
\usepackage{verbatim}

\usepackage{amssymb}
\usepackage{hyperref}       % hyperlinks
\usepackage{subcaption}

\title{Rapport de stage : GAN}

\author{Lucas Goareguer \\
  Aix-Marseille Université \\
  \texttt{Lucas.Goareguer@etu.univ-amu.fr   } \\\And
   {\bf Supervision : Laurent Perrinet} \\
  Institut de Neurosciences de la Timone \\
  {\tt Laurent.Perrinet@univ-amu.fr} \\}
\date{}

%%%%%%%%%%%%%%%%%%%%%%%%%%%%%

\begin{document}
\maketitle
\begin{abstract}
Rapide résumer du cadre du stage et de ses objectifs.
Ce papier à pour objectif de décrire le travail effectuer durant le stage terminal du Master Intelligence Artificielle et Apprentissage Automatique.
Ce stage de 6 mois c'est dérouler à l'Institut de Neurosciences de la Timone et a était encadrer par Laurent Perrinet.
Le principal objectif du stage était la compréhension des Generative Adverserials Networks (GAN) appliquer à la génération d'images.
\end{abstract}

\blfootnote{
This work is licensed under a Creative Commons 
Attribution 4.0 International License.
License details:  \url{http://creativecommons.org/licenses/by/4.0/}
}

%%%%%%%%%%%%%%%%%%%%%%%%%%%%%
\section{Motivation}

\subsection{Pourquoi les GANs}
% https://stackoverflow.com/questions/50131068/producing-stacked-3d-blocks-using-tikz

\subsection{Compréhension des limites théoriques de ces modèles}

\subsection{Adverserial Auto-Encoder}
% http://www.texample.net/tikz/examples/simple-flow-chart/

\subsection{Étude de l'espace latent}
lien avec VAE de Kingma 
%%%%%%%%%%%%%%%%%%%%%%%%%%%%%
\section{Méthodes}

\subsection{Strange Attractor}
\subsection{Simpsons Dataset}
\subsection{Scan des paramètres}

%%%%%%%%%%%%%%%%%%%%%%%%%%%%%
\section{Résultats}

\subsection{Simpsons Cohérent}
\subsection{Évolution du loss de G (convergence, équilibre de Nash,..)}
\subsection{Comparaison des espaces latent connu et générer}
\subsection{Interpolation dans l'espace latent}

%%%%%%%%%%%%%%%%%%%%%%%%%%%%%
\section{Conclusions et perspectives}

\subsection{Apprentissage des GANs}
\subsection{Découverte et prise en mains de nombreux outils}
\subsection{Premier pas dans le monde professionnels}

\newpage
%{\bf Appendices}: Appendices, if any, directly follow the text and the references (but see above).  Letter them in sequence and provide an informative title: {\bf Appendix A. Title of Appendix}.
%\section*{Acknowledgements}

%The acknowledgements should go immediately before the references.  Do
%not number the acknowledgements section. Do not include this section
%when submitting your paper for review.

\section*{Remerciements}
Je tiens à remercier Laurent Perrinet pour m'avoir encadrés et guidés durant ce Stage.

\bibliography{biblio}
\bibliographystyle{acl}
%%%%%%%%%%%%%%%%%%%%%%%%%%%%
\newpage
\section*{Annexe 1. Description des données}

%%%%%%%%%%%%%%%%%%%%%%%%%%%%


\end{document}
